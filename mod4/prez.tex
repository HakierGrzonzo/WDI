\documentclass[aspectratio=169]{beamer}

\usepackage{inputenc}[utf8]
\usepackage[polish]{babel}

%Lepiej tego nie zmieniaj, jak co to dodawaj pakiety
\usepackage{fancyhdr}
\usepackage{mdframed}
\usepackage{graphicx}
\usepackage{listings}
\usepackage{caption}
\usepackage{float}
\usepackage{hyperref}
\hypersetup{
    colorlinks=false,
    linkcolor=blue,
    filecolor=magenta,      
    urlcolor=cyan,
}
%\apptocmd{\frame}{}{\justifying}{} % Allow optional arguments after frame.

\urlstyle{same}
%Zmienne, zmień je!
\graphicspath{ {./ilustracje/} }
\title[docker oraz docker-compose 101]{Tworzenie i uruchamianie prostych kontenerów w środowisku  docker-compose}
\author{Grzegorz Koperwas}
\date{\today}

%lokalizacja polska (odkomentuj jak piszesz po polsku)

\usepackage{polski}
\usepackage[polish]{babel} 
\usepackage{indentfirst}
\usepackage{icomma} 
\usetheme{Warsaw}

\brokenpenalty=1000
\clubpenalty=1000
\widowpenalty=1000    

%nie odkometowuj wszystkiego, użyj mózgu
%\renewcommand\thechapter{\arabic{chapter}.}
\renewcommand\thesection{\arabic{section}.}
\renewcommand\thesubsection{\arabic{section}.\arabic{subsection}.}
\renewcommand\thesubsubsection{\arabic{subsubsection}.}

%Makra

\newcommand{\obrazek}[2]{
\begin{figure}[h]
    \centering
    \includegraphics[scale=#1]{#2}
\end{figure}
}     
        

\newcommand{\twierdzonko}[1]{
    \begin{center}
    \begin{mdframed}
    #1
    \end{mdframed}          
    \end{center}
} 

\newcommand{\dwanajeden}[2]{
\ensuremath \left( \begin{array}{c}
    #1\\
    #2
\end{array} \right)
}  

\captionsetup[figure]{name=Załącznik}
\begin{document}
\begin{frame}
    \titlepage
\end{frame}
\begin{frame}
    \tableofcontents
\end{frame}
\section{docker 101}
\subsection{Po co docker}
\begin{frame}
    \frametitle{Po co jest docker?}

    \begin{columns}
        \begin{column}{0.5\textwidth}
            \texttt{Docker} znacząco ułatwia instalowanie i zarządzanie oprogramowaniem w systemach \emph{Linux}.
            \pause
            \begin{itemize}
                \item Izoluje programy oraz ich zależności.
                \pause
                \item Łatwa instalacja programów spoza repozytoriów naszej dystrybucji.
                \pause
                \item Latwe zarządzanie obrazami aplikacji.
            \end{itemize}
        \end{column}
        \begin{column}{0.5\textwidth}
            \pause
            \begin{itemize}
                \item Skalowalność dla wielu serwerów za pomocą \emph{Kubernetes}
            \end{itemize}
            \pause[0]
            \obrazek{0.27}{docker-arch.png}
        \end{column}
    \end{columns}
\end{frame}
\begin{frame}
    \frametitle{Do czego nie jest docker}
    \begin{columns}
        \begin{column}{0.5\textwidth}
            Docker ma oczywiście swoje wady:
            \pause
            \begin{itemize}
                \item Nie jest tak szybki jak jego brak.
                \pause
                \item Trwałe przechowywanie danych jest \emph{ciekawe.}
            \end{itemize}
            \pause
            Jeśli coś może być łatwo osiągnięte czymś takim jak \texttt{systemd}, to warto to tak zrobić.
        \end{column}
        \begin{column}{0.5\textwidth}
            \obrazek{0.15}{systemD.png}
        \end{column}
    \end{columns}
\end{frame}
\subsection{Po co obrazy aplikacji}
\begin{frame}
    \frametitle{Czym jest obraz aplikacji i plik dockerfile}
    By uruchomić daną aplikację w środowisku \texttt{docker}'a musimy utworzyć jej obraz.
    \pause

    \hspace{5mm}

        \texttt{Docker} tworzy swoje obrazy według instrukcji w pliku zwanym \texttt{dockerfile}.
        
    \pause
    \hspace{5mm}

        Możemy o nich myśleć jako instrukcjach które znajdują się na wielu repozytoriach na \emph{Github'ie} mówiących jakie biblioteki są potrzebne do kompilacji oraz jak kompilować i instalować dany program.
\end{frame}
\begin{frame}
    Na potrzeby tej prezentacji będziemy chcieli stworzyć obraz z kompilatorem sieciowym \emph{distcc}.

    \pause
    \vspace{5mm}
    Pierwszym krokiem do stworzenia jakiegoś obrazu aplikacji musimy wybrać obraz \emph{bazowy}.

    \pause
    \vspace{5mm}
    My będziemy wykorzystywali obraz \texttt{\href{https://hub.docker.com/_/archlinux/}{archlinux:latest}}.
    \pause
    \begin{block}{Nazwy obrazów}
        Nazwy obrazów są w formie:
        \vspace{-3mm}
        \begin{center}
            \texttt{<autor>?/<nazwa>:tag}
        \end{center}
        Na przykład:
        \vspace{-3mm}
        \begin{center}
            \href{https://hub.docker.com/r/jrottenberg/ffmpeg/}{\texttt{jrottenberg/ffmpeg:4.1-nvidia}}
        \end{center}
    \end{block}
\end{frame}
\subsection{Pliki dockerfile}
\begin{frame}[fragile]
    \frametitle{Plik Dockerfile}

    \begin{figure}
        \begin{lstlisting}[frame=L,basicstyle=\tiny\ttfamily,numbers=left,
        morekeywords={FROM,RUN,EXPOSE,ENTRYPOINT}]
FROM archlinux:latest
RUN pacman -Syu --noconfirm distcc make git gcc cmake
COPY ./dummy.txt /etc/dummy.txt
EXPOSE 3632/tcp
EXPOSE 3632/udp
ENTRYPOINT distccd --daemon --no-detach --verbose --allow-private
    \end{lstlisting}
    \end{figure}
    Plik \texttt{dockerfile składa się z takich instrukcji:}
    \pause
    \begin{description}
        \item[FROM] Definiuje bazowy obraz
    \pause
        \item[RUN] Uruchamia komendę w kontenerze
    \pause
        \item[EXPOSE] Udostępnia porty w kontenerze
    \pause
        \item[COPY] Kopiuje plik/katalog \emph{A} z naszego komputera do katalogu \emph{B} w obrazie.
    \end{description}
\end{frame}
\begin{frame}[fragile]
    \frametitle{Plik Dockerfile}
    Obraz budujemy za pomocą polecenia:

    \begin{figure}[H]
        \begin{lstlisting}[frame=single,basicstyle=\footnotesize\ttfamily,language=bash,morekeywords={docker}]
$ docker build -t $nazwaObrazu .
    \end{lstlisting}
    \end{figure}

    Gdzie zamiast \texttt{\$nazwaObrazu} wpisujemy nazwę dla naszego obrazu.
\end{frame}

\section{Uruchamianie w docker-compose}
\begin{frame}[fragile]
    \frametitle{Uruchamianie obrazów w środowisku \texttt{docker-compose}}
    \begin{columns}
        \begin{column}{0.4\textwidth}
            Domyślnie \texttt{Docker} uruchamia obrazy poleceniami \texttt{docker~run}.
    \pause

            \vspace{5mm}

            \texttt{docker-compose} ułatwia proces uruchamiania poprzez zastosowanie pliku \emph{YAML} z opcjami które normalnie byśmy wpisywali jako argumenty polecenia \texttt{docker~run}
        \end{column}
        \begin{column}{0.6\textwidth}
    \pause[0]
            \begin{figure}[H]
                \begin{lstlisting}[frame=single,basicstyle=\scriptsize\ttfamily,morekeywords={docker}]
# docker run \
    --gpus all \
    --network "host" \
    --device /dev/ttyUSB0:/dev/tty2 \
    --volume /bar/off/foo:/bar \
    rtsp_over_serial:latest
    \end{lstlisting}
            \end{figure}
        \end{column}
    \end{columns}
\end{frame}

\subsection{Plik docker-compose.yml}
\begin{frame}[fragile]
    \frametitle{Format pliku \texttt{docker-compose.yml}}
    \begin{columns}
        \begin{column}{0.45\textwidth}
            \begin{lstlisting}[frame=L,basicstyle=\scriptsize\ttfamily,numbers=left,
        morekeywords={version,services,image,ports,restart}]
version: '3.3'

services:
    distcc:
        image: distcc:latest
        restart: always
        ports:
            - "3632:3632"
    \end{lstlisting}
            Plik jest w formacie \emph{YAML}, w formie \texttt{klucz~$\rightarrow$~wartość}
    \pause

            \vspace{3mm}

            Pod kluczem \texttt{services} umieszczamy kontenery jakie będziemy uruchamiać.
        \end{column}
        \begin{column}{0.55\textwidth}
    \pause
            Kolejne klucze określają opcje dla kontenera (i z jakiego obrazu go stworzyć)

            \small{
                \begin{description}
                    \item[image] Nazwa obrazu dla kontenera.
    \pause
                    \item[restart] Czy restartować kontener po jego zakończeniu.
    \pause
                    \item[ports] Konfiguracja forwardowania portów, w formie \texttt{<port hosta>:<port kontenera>}
                \end{description}}
        \end{column}
    \end{columns}
\end{frame}
\subsection{Uruchamianie kontenerów za pomocą docker-compose}
\begin{frame}[fragile]
    \frametitle{Uruchamianie kontenerów komendą docker-compose}
    By uruchomić kontener wystarczy polecenie:
    \begin{lstlisting}[frame=single,basicstyle=\footnotesize\ttfamily,language=bash,morekeywords={docker-compose}]
$ docker-compose up -d
    \end{lstlisting}
    Gdzie opcja \texttt{-d} odłącza proces od terminala.
    \pause
    \begin{block}{Bonus}
        \texttt{Docker-compose} udostępnia nam też inne komendy takie jak \texttt{docker-compose logs} wypisującą logi na konsoli. Za pomocą takiego one-linera możemy sobie je wyświetlać na bieżąco:
        \begin{center}\begin{minipage}{0.9\textwidth}
        \begin{lstlisting}[frame=single,basicstyle=\scriptsize\ttfamily,morekeywords={tail,docker-compose,watch}]
$ watch --color "docker-compose logs | tail -n 20"
    \end{lstlisting}
        \end{minipage}\end{center}
    \end{block}
\end{frame}
\end{document}
