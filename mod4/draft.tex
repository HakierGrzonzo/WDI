% !TEX program = xelatex
%Wzór dokumentu
%tu zmień marginesy i rozmiar czcionki
\documentclass[a4paper,12pt]{article}
\usepackage{inputenc}[utf8]
\usepackage[margin=2.8cm]{geometry}
\usepackage[polish]{babel}

%Lepiej tego nie zmieniaj, jak co to dodawaj pakiety
\usepackage{titlesec}
\usepackage{titling}
\usepackage{fancyhdr}
\usepackage{mdframed}
\usepackage{graphicx}
\usepackage{amsmath}
\usepackage{amsfonts}
\usepackage{multicol}
\usepackage{listings}
\usepackage{caption}
\usepackage{float}



%inny wygląd
%\usepackage{tgbonum}


\usepackage{hyperref}
\hypersetup{
    colorlinks=true,
    linkcolor=blue,
    filecolor=magenta,      
    urlcolor=cyan,
}

\urlstyle{same}
%Zmienne, zmień je!
\graphicspath{ {./ilustracje/} }
\title{Tworzenie i uruchamianie prostych kontenerów w środowisku  docker-compose}
\author{Grzegorz Koperwas}
\date{\today}

%lokalizacja polska (odkomentuj jak piszesz po polsku)

\usepackage{polski}
\usepackage[polish]{babel} 
\usepackage{indentfirst}
\usepackage{icomma} 

\brokenpenalty=1000
\clubpenalty=1000
\widowpenalty=1000    

%nie odkometowuj wszystkiego, użyj mózgu
%\renewcommand\thechapter{\arabic{chapter}.}
\renewcommand\thesection{\arabic{section}.}
\renewcommand\thesubsection{\arabic{section}.\arabic{subsection}.}
\renewcommand\thesubsubsection{\arabic{subsubsection}.}

%Makra

\newcommand{\obrazek}[2]{
\begin{figure}[h]]
    \centering
    \includegraphics[scale=#1]{#2}
\end{figure}
}     
        

\newcommand{\twierdzonko}[1]{
    \begin{center}
    \begin{mdframed}
    #1
    \end{mdframed}          
    \end{center}
} 

\newcommand{\dwanajeden}[2]{
\ensuremath \left( \begin{array}{c}
    #1\\
    #2
\end{array} \right)
}  

\captionsetup[figure]{name=Załącznik}
%Stopka i head (sekcja której nie powinno się zmieniać)
\pagestyle{fancy}
\fancyhead{}
\fancyfoot{}

%Zmieniaj od tego miejsca
\rfoot{\thepage}
\lfoot{}
\lhead{}
\rhead{Ostatnia edycja: \today}
\renewcommand{\headrulewidth}{1pt}
\renewcommand{\footrulewidth}{1pt}

\begin{document}
\maketitle
\thispagestyle{fancy}

\begin{abstract}
    W tej pracy zostaną pokazane absolutne podstawy pisania plików \texttt{dockerfile} oraz uruchamiania tych kontenerów w środowisku \texttt{docker-compose} na serwerze z systemem \emph{GNU/Linux}. Autor zakłada iż czytelnik posiada zainstalowanego \texttt{docker}'a oraz \texttt{docker-compose} w miarę aktualnych wersjach.
\end{abstract}


\section{Jak działają obrazy aplikacji?}
\nocite{hightower2017kubernetes}
By uruchomić daną aplikację w środowisku \texttt{docker}'a musimy utworzyć jej obraz. \texttt{Docker} tworzy swoje obrazy według instrukcji w pliku zwanym \texttt{dockerfile}, możemy o nich myśleć jako instrukcjach które znajdują się na wielu repozytoriach na \emph{Github'ie} mówiących jakie biblioteki są potrzebne do kompilacji oraz jak kompilować i instalować dany program, czasami nawet zawierają one informacje dla poszczególnych dystrybucji.

Na potrzeby tej pracy będziemy chcieli stworzyć obraz z kompilatorem sieciowym \emph{distcc\footnote{\url{https://distcc.github.io/}}}.

Pierwszym krokiem do stworzenia jakiegoś obrazu aplikacji musimy wybrać obraz \emph{bazowy}. Docker posiada swoją platformę \emph{Docker Hub} na której możemy szukać właśnie interesujących nas obrazów, które mają formę od gotowych do pracy dystrybucji linuxa (alpine, arch czy debian) do gotowych obrazów popularnych aplikacji wraz z sterownikami graficznymi\footnote{Na przykład \texttt{\href{https://hub.docker.com/r/jrottenberg/ffmpeg/}{jrottenberg/ffmpeg:4.1-nvidia}} zawiera debian'a z programem ffmpeg skonfigurowanym tak by wykorzystywał akceleracje wideo kart firmy \emph{nvidia}.}.

My będziemy wykorzystywali obraz \texttt{\href{https://hub.docker.com/_/archlinux/}{archlinux:latest}}, nie dlatego że jest to najlepszy wybór, ale dlatego iż możemy łatwo go skonfigurować. Warto tu jeszcze powiedzieć o \emph{tagach} (ta część po ,,:''), w naszym przypadku używamy taga \emph{latest}, czyli najnowszej wersji obrazu, jednak w wypadku innych obrazów należy sumiennie przeczytać dokumentację w celu wyboru właściwego.

\subsection*{Co napisać w pliku \texttt{dockerfile}?}

\begin{figure}[p]
    \begin{lstlisting}[frame=L,basicstyle=\footnotesize\ttfamily,numbers=left,
        morekeywords={FROM,RUN,EXPOSE,ENTRYPOINT}]
FROM archlinux:latest
RUN pacman -Syu --noconfirm distcc make git gcc cmake
EXPOSE 3632/tcp
EXPOSE 3632/udp
ENTRYPOINT distccd --daemon --no-detach --verbose --allow-private
    \end{lstlisting}
    \caption{Dockerfile tworzący nasz obraz z programem \emph{distcc}}
    \label{dockerfile-comp}
\end{figure}

Całość zawartości pliku dockerfile jest przedstawiona w załączniku \ref{dockerfile-comp}, omówmy zatem co się dzieje w każdej linijce jego zawartości.

\subsubsection{Polecnie \texttt{FROM}}

Polecenie \texttt{FROM} mówi \emph{docker}'owi z jakiego obrazu ma on stwożyć nasz obraz, czyli definiujemy obraz bazowy. Widzimy iż przekazujemy mu nazwę obrazu \texttt{archlinux:latest}, jeśli nie mamy jej jeszcze na swoim komputerze to \emph{docker} pobierze ją z \emph{docker hub}'a

\subsubsection{Polecenie \texttt{RUN}}

Polecenie \texttt{RUN} pozwala nam uruchomić dane polecenie w kontenerze (\emph{docker} uruchomi obraz, wykona polecenie, zapisze obraz). Często używamy go do instalowania jakiś zależności lub jak tutaj, aplikacji którą chcemy ,,skonteneryzować''.

Polecenie \texttt{pacman -Syu --noconfirm distcc}[\dots]\texttt{cmake} aktualizuje wszystkie pakiety oraz instaluje pakiety distcc, make, git, gcc i cmake bez pytania się użytkownika o potwierdzenie (\texttt{--noconfirm})
\vspace{5mm}
\begin{center}
    \begin{minipage}{14cm}
        \twierdzonko{
            \begin{center}
                \large{Ważne!}
            \end{center}
            \begin{enumerate}
                \item Docker uruchamia wszystkie polecenia w kontenerach jako \texttt{root}
                \item Docker by zwiększyć prędkość tworzenia obrazów zapisuje je do cache po każdym kroku, zatem warto je łączyć (patrz załącznik \ref{run-connect})
            \end{enumerate}
        }
    \end{minipage}
\end{center}

\begin{figure}[p]
    \begin{center}
        \begin{minipage}{4cm}
            Zamiast:
            \begin{lstlisting}[frame=L,basicstyle=\footnotesize\ttfamily,
                morekeywords={RUN},numbers=left,firstnumber=2]
RUN command1
RUN command2
            \end{lstlisting}
        \end{minipage}
        \hspace{5mm}
        \begin{minipage}{4cm}
            Lepiej:
            \begin{lstlisting}[frame=L,basicstyle=\footnotesize\ttfamily,
                morekeywords={RUN},numbers=left,firstnumber=2]
RUN command1 && \
    command2
            \end{lstlisting}
        \end{minipage}
    \end{center}
    \vspace{-7mm}
    \caption{Lepiej wiele poleceń dać do jednego \emph{RUN'a}}
    \label{run-connect}
\end{figure}

\subsubsection{Polecenia \texttt{EXPOSE}}

Polecenie \texttt{EXPOSE} pozwala nam otworzyć dane porty w kontenerze dla danych protokołów. Na przykład \texttt{EXPOSE 3632/tcp} otwiera port \emph{distcc} dla protokołu tcp, a \texttt{EXPOSE 3632/udp} dla protokołu udp.

\subsubsection{Polecenie \texttt{ENTRYPOINT}}

Poleceniem \texttt{ENTRYPOINT} definiujemy jaki program lub skrypt powinien zostać uruchomiony w kontenerze kiedy zaczyna on swoją pracę. Warto zaznaczyć że docker zakłada iż kontener skończył swoją pracę jeśli polecenie skończy się wykonywać, dlatego dodajemy opcję \texttt{--no-detach} do polecenia uruchamiającego daemona \emph{distccd}.

Opcja \texttt{--allow-private} udostępnia kompilację dla wszystkich klientów w sieci lokalnej.

\subsubsection{Polecenie \texttt{COPY}}

Warto również wspomnieć o poleceniu \texttt{COPY}, które nie jest wykorzystywane w naszym \texttt{dockerfile}'u ale jest przydatne jeśli chcemy przekopiować jakiś nasz lokalny plik do obrazu.

Przykładowo w drugiej linijce załącznika \ref{dockerfile-portlister} kopiujemy folder z aplikacją \texttt{./app}\footnote{Ścieżka jest podana relatywnie do lokalizacji pliku \texttt{dockerfile}} do folderu \texttt{/app} w obrazie.

\begin{figure}[p]
    \begin{lstlisting}[frame=L,basicstyle=\footnotesize\ttfamily,numbers=left,
        morekeywords={FROM,COPY,RUN}]
FROM tiangolo/uwsgi-nginx-flask:python3.8
COPY ./app /app
RUN python3 -m pip install -r /app/requirements.txt
    \end{lstlisting}
    \caption{Dockerfile tworzący obraz mojej webowej aplikacji \emph{PortLister} w \texttt{flask}'u}

    \label{dockerfile-portlister}
\end{figure}

\vspace{1cm}

Istnieją również inne polecenia których możemy użyć w \texttt{dockerfile}, są one omówione w dokumentacji\cite{dockerfile:docs}.

\subsection*{Budowanie obrazu z pliku \texttt{dockerfile}}

Obraz budujemy za pomocą polecenia:

\begin{figure}[H]
    \begin{lstlisting}[frame=single,basicstyle=\footnotesize\ttfamily,language=bash,morekeywords={docker}]
$ docker build -t $nazwaObrazu .
    \end{lstlisting}
\end{figure}

Gdzie zamiast \texttt{\$nazwaObrazu} wpisujemy nazwę dla naszego obrazu.

\section{Uruchamianie obrazów w środowisku \texttt{docker-compose}}

\texttt{Docker-compose} jest alternatywą dla zwykłego uruchamiania kontenerów poprzez polecenia \texttt{docker run} gdzie bardzo szybko mogą nam powstać takie cuda:

\begin{figure}[H]
    \begin{lstlisting}[frame=single,basicstyle=\footnotesize\ttfamily,morekeywords={docker,\\},numbers=left]
# docker run \
    --gpus all \
    --network "host" \
    --device /dev/ttyUSB0:/dev/tty2 \
    --volume /dockerStuff/rtsp/data:/rtsp/data \
    rtsp_over_serial:latest
    \end{lstlisting}
\end{figure}

Zapamiętanie takich monstrów\footnote{Przedstawiony przykład i tak jest rozmiaru ,,średniego''} na rzecz późniejszej edycji możemy jedynie powierzyć historii konsoli. Dlatego możemy używać środowiska \texttt{docker-compose} w celu uruchamiania kontenerów za pomocą ustawień w pliku \texttt{docker-compose.yml}. Plik ten używa formatu \emph{YAML}.

Plik dla naszego obrazu znajduje się w załączniku \ref{dockercompose}, omówmy sobie jego zawartość.

\subsection*{Tworzenie plilku \texttt{docker-compose.yml}}

W pliku \texttt{docker-compose.yml} pod kluczem ,,services'' umieszczamy klucze z nazwami jakie chcemy nadać naszym kontenerom. Nasz kontener z kompilatorem \emph{distcc} nazywamy oczywiście distcc.

\subsubsection{Klucz \texttt{image}}

W tym kluczu umieszczamy nazwę naszego obrazu, w naszym wypadku jest to ,,distcc'' wraz z tagiem ,,latest'', który wybiera najnowszą wersję.

\subsubsection{Klucz \texttt{restart}}

Klucz \texttt{restart} pozwala nam przekazać środowisku \texttt{docker-compose} co ma się dziać po zakończeniu pracy kontenera lub przy restartowaniu serwera. Dostępnych mamy parę opcji:
\begin{itemize}
    \item \emph{always} - kontener jest zawsze uruchamiany ponownie. Dlatego wybieramy go dla naszego kontenera.
    \item \emph{on-failure} - nie restartuje kontenera jeśli zwróci on 0.
    \item \emph{no} - nie restartuje kontenera nigdy - domyślny.
\end{itemize}

\subsubsection{Klucz \texttt{ports}}

Pod tym kluczem możemy udostępniać porty z kontenera, w formie \texttt{<port hosta>:<port kontenera>}.

\vspace{1cm}

Istnieje o wiele więcej kluczy których możemy używać, ich opis możemy łatwo znaleźć w dokumentacji \texttt{docker-compose}\cite{docker-compose:docs}.

\subsection*{Uruchamianie kontenera}

By uruchomić kontener wystarczy polecenie:

\begin{figure}[H]
    \begin{lstlisting}[frame=single,basicstyle=\footnotesize\ttfamily,language=bash,morekeywords={docker-compose}]
$ docker-compose up -d
    \end{lstlisting}
\end{figure}

Gdzie opcja \texttt{-d} odłącza proces od konsoli.

\subsubsection*{Bonusowy one-liner}
By wyświetlać logi na bieżąco polecam polecenie:
\begin{figure}[H]
    \begin{lstlisting}[frame=single,basicstyle=\footnotesize\ttfamily,language=bash,morekeywords={tail,docker-compose,watch}]
$ watch --color "docker-compose logs | tail -n 20"
    \end{lstlisting}
\end{figure}


\begin{figure}[p]
    \begin{lstlisting}[frame=L,basicstyle=\footnotesize\ttfamily,numbers=left,
        morekeywords={version,services,image,ports,restart}]
version: '3.3'

services:
        distcc:
                image: distcc:latest
                restart: always
                ports:
                        - "3632:3632"
    \end{lstlisting}
    \caption{Plik \texttt{docker-compose.yml} dla naszego obrazu z \emph{distcc}}
    \label{dockercompose}
\end{figure}

\bibliographystyle{alpha}
\bibliography{draft}
\end{document}
