\documentclass{beamer}
	\usepackage{inputenc}[utf8]
	\usepackage{mdframed}
	\usepackage{graphicx}
	\usepackage{amsmath}
	\usepackage{amsfonts}
    \usepackage{polski}
    \usepackage[polish]{babel} 
    \usepackage{indentfirst}
    \usepackage{icomma}
    \usepackage{minted}
    \usepackage{ragged2e}
    \brokenpenalty=1000
    \clubpenalty=1000
    \widowpenalty=1000    
    \apptocmd{\frame}{}{\justifying}{} % Allow optional arguments after frame.
    \newcommand{\obrazek}[2]{
        \begin{figure}[h]
            \centering
            \includegraphics[scale=#1]{#2}
        \end{figure}
    }     
            
    
    \newcommand{\twierdzonko}[1]{
        \begin{center}
        \begin{mdframed}
        #1
        \end{mdframed}          
        \end{center}
    } 
    \usetheme{Warsaw}
    \title{Algorytmy}
    \author{Grzegorz Koperwas \and Mateusz Randak \and Piotr Kołodziejski}
    \date{\today}

\begin{document}
\begin{frame}
    \titlepage
\end{frame}

\section{Czym jest algorytm?}
\begin{frame}
    \frametitle{Czym jest algorytm?}
    
    \begin{block}{Definicja}
        \emph{Algorytm} to zestaw instrukcji opisujących krok po kroku jak wykonać pewne zadanie.
    \end{block}
    
   Z algorytmami mamy do czynienia w codziennym życiu wykonując różne czynności według pewnych schematów. Przykładowo gdy gotujemy i podążamy za jakimś przepisem.
   
   Algorytmy można przedstawiać w różny sposób, najczęściej przedstawiane są w formie listy kroków lub schematu blokowego.
\end{frame}

\section{Czym jest algorytm?}
\begin{frame}
     \frametitle{Przykład}
     
    \obrazek{.5}{lista-i-schemat.jpg}
    \end{block}
    
\end{frame}

\section{Czym jest algorytm?}
\begin{frame}
     \frametitle{Algorytmy w informatyce}

    Algorytmy odgrywają niezwykle ważną rolę w informatyce. Aby oprogramowanie działało najefektywniej, konieczne jest wybranie algorytmu o jak najmniejszej złożoności czasowej. 
    
     \begin{block}{Definicja}
        Złożoność czasowa - Jak długo zajmuje wykonanie algorytmu w zależności od wielkości wejścia $\left(n\right)$
    \end{block}
    
	Dobrym przykładem jest algorytm wyszukiwania binarnego, którego czas wykonania może być znacznie krótszy od algorytmu prostego wyszukiwania, szczególnie w przypadku dużych zbiorów.

\end{frame}

\section{Wyszukiwanie binarne}
\begin{frame}
    \frametitle{Wyszukiwanie binarne:}
    \emph{Wyszukiwanie binarne} - algorytm stosowany do wyszukiwania
    elementów w posegregowanych zbiorach.

    Cechuje się złożonością czasową $O \left(\log_2 n \right)$, podczas gdy wyszukiwanie liniowe (element po elemencie), ma złożoność $O \left( n \right)$.

   
\end{frame}

\begin{frame}
    \frametitle{Zasada działania}
    Główną ideą wyszukiwania binarnego jest podział elementów w przeszukiwanym uporządkowanym zbiorze danych
    (np. tablicy) na coraz to mniejsze zbiory, tak by optymalnie ograniczyć zakres poszukiwania.
    %Należy zaznaczyć, że elementy muszą być ze sobą \emph{porównywalne}, tzn. spośród dowolnych dwóch elementów
    %da się wyłonić większy i mniejszy (ew.~równy), np. po wartości lub nazwie.

    \begin{enumerate}
        \item gdy środkowy element jest mniejszy od szukanej, za dolną granicę uznaje się kolejny element, a górna pozostaje bez zmian
        \item gdy środkowy element jest większy od szukanej, za górną granicę uznaje się poprzedni element,
              a dolna pozostaje bez zmian
    \end{enumerate}

    Następnie powyższe kroki zostają powtórzone. Jeżeli element nie zostanie znaleziony,
    zwracana jest odpowiednia wartość wskazująca na błąd (np. $-1$ lub $null$).
\end{frame}

\begin{minted}[fontsize=\small]{python}
def binary_search(item, collection):
"""searches for specified item in chosen collection"""
    lower_bound = 0
    upper_bound = len(collection) - 1

    while lower_bound <= upper_bound:
        mid = (lower_bound + upper_bound) // 2

        if collection[mid] == item:
            return mid

        if collection[mid] < item:
            lower_bound = mid + 1
        else:
            upper_bound = mid - 1

    # if item not found
    raise Exception("not found")
\end{minted}

\begin{frame}
    \frametitle{Przykład:}
    \obrazek{.5}{binarySearchExample.png}
\end{frame}

\section{Quicksort}
\begin{frame}
    \frametitle{Quicksort:}
    Quicksort to rekurencyjny algorytm sortujący, działający poprzez dzielenie początkowego zbioru na coraz mniejsze podzbiory.
    \vspace{5mm}

    Działanie:
    \begin{enumerate}
        \item Wybierz element z \emph{danego zbioru}, zwany \emph{elementem rozdzielającym.}
        \item Elementy mniejsze od \emph{elementu rozdzielającego} przenieś do zbioru $A$, większe do $C$, a równe do $B$.
        \item Posortuj zbiory $A$ i $C$ quicksortem.
        \item Wynikiem jest połączenie zbiorów $A + B + C$.
    \end{enumerate}
\end{frame}

\begin{minted}[fontsize=\small]{python}
def quicksort(zbiór):
    if len(zbiór) <= 1:
        # nie trzeba sortować zbiorów pustych 
        return zbiór
    else:
        element_rozdzielający = zbiór[-1] # ostatni element
        a = list() # mniejsze
        b = list() # równe
        c = list() # większe
        for element in zbiór:
            if element < element_rozdzielający:
                a.append(element)
            elif element > element_rozdzielający:
                c.append(element)
            else:
                b.append(element)
        a = quicksort(a)
        c = quicksort(c)
        return a + b + c
\end{minted}

\begin{frame}
    \frametitle{Przykład:}
    \obrazek{.5}{quicksortExample.png}
\end{frame}

\begin{frame}
    \frametitle{Koniec}
    \begin{center}
        Dziękujemy za uwagę!
    \end{center}
\end{frame}
\end{document}
